\documentclass[conference]{IEEEtran}
\usepackage[utf8]{inputenc}
\IEEEoverridecommandlockouts

\usepackage{cite}
\usepackage{amsmath,amssymb,amsfonts}
\usepackage{algorithmic}
\usepackage{algorithm}
\usepackage{graphicx}
\usepackage{textcomp}
\graphicspath{{./images/}}
\usepackage{xcolor}

\def\BibTeX{{\rm B\kern-.05em{\sc i\kern-.025em b}\kern-.08em
T\kern-.1667em\lower.7ex\hbox{E}\kern-.125emX}}

\begin{document}

\title{Conference Paper Title\\}

\author{
\IEEEauthorblockN{Anusha Adarakati}
\IEEEauthorblockA{
\textit{KLE Technological University} \\
Hubli, India \\
01fe22bcs186@kletech.ac.in}
\and
\IEEEauthorblockN{Anirudh Dambal}
\IEEEauthorblockA{
\textit{KLE Technological University} \\
Hubli, India \\
01fe22bcs171@kletech.ac.in}
\and
\IEEEauthorblockN{Pavan Bhakta}
\IEEEauthorblockA{
\textit{KLE Technological University} \\
Hubli, India \\
01fe22bcs175@kletech.ac.in}
\and
\IEEEauthorblockN{Harish Patil}
\IEEEauthorblockA{
\textit{KLE Technological University} \\
Hubli, India \\
01fe22bcs173@kletech.ac.in}
\and
\IEEEauthorblockN{Professor Heisenburger}
\IEEEauthorblockA{
\textit{KLE Technological University} \\
Hubli, India \\
nigga@kletech.ac.in}
}

\maketitle


\begin{abstract}
AI-generated music is now much better thanks to recent developments in deep generative models. Nevertheless, little is known about the production and controllability of many musical styles. We present Muse Netic, a dual-architecture framework for autonomous music generation in this study that aims for stylistic authenticity in both jazz and classical genres. Muse Netic combines a waveform-based Convolutional Variational Autoencoder (CVAE) for jazz synthesis with an LSTM-based symbolic model for creating Chopin-style classical piano music. While the CVAE uses 1D ResNet blocks to encode and reconstruct raw audio waveforms from the GTZAN jazz dataset, the LSTM records long-term dependencies in symbolic MIDI sequences. We present a targeted data sampling and preprocessing approach that maintains genre-defining audio qualities including rhythm, texture, and timbre in order to overcome the problem of a lack of high-quality training samples. Extensive trials show that Muse Netic surpasses standalone LSTMs and standard VAEs in terms of reconstruction quality and style consistency, and produces expressive, genre-faithful music.
\end{abstract}

\begin{IEEEkeywords}
keywords
\end{IEEEkeywords}

\section{Introduction}
This is the introduction.

\section{Methodology}
Description of the methodology.

\section{Results and Discussion}
Discussion of the results.

\section{Conclusion}
Conclusion of the paper.

\section*{Acknowledgment}
Acknowledgment of contributors or funding.

\begin{thebibliography}{00}
\bibitem{b1} Author, ``Title of the paper,'' \textit{Journal Name}, vol. X, no. Y, pp. Z--ZZ, Year.
\bibitem{b2} Author, \textit{Book Title}, Edition, Publisher, Year.
\end{thebibliography}

\end{document}
